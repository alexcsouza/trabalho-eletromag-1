\documentclass{beamer}

%encoding
%--------------------------------------
\usepackage[T1]{fontenc}
\usepackage[utf8]{inputenc}
%--------------------------------------

%Portuguese-specific commands
%--------------------------------------
\usepackage[portuguese]{babel}
%--------------------------------------

%Hyphenation rules
%--------------------------------------
\usepackage{hyphenat}
\hyphenation{mate-mática recu-perar}
%--------------------------------------

\usepackage{wrapfig}
\usepackage{graphicx}
\usepackage{biblatex}

% \usepackage[
% backend=biber,
% style=alphabetic,
% sorting=ynt
% ]{biblatex}
\addbibresource{latex-beamer-template.bib}

\graphicspath{ {../../images/} {.images} }

% \usefonttheme{structuresmallcapsserif}
\usetheme{Warsaw}
% \usecolortheme{beetle}
\usecolortheme{seagull}

% \setbeamertemplate{navigation symbols}{}
% \setbeamertemplate{mini frames}{}
\setbeamertemplate{headline}{}
% \renewcommand*{\slideentry}[6]{}
\setbeamertemplate{section in toc}[sections numbered]
\setbeamertemplate{subsection in toc}[subsections numbered]

\setbeamercolor{block body alerted}{bg=yellow!10, bg=alerted text.fg!10}
\setbeamercolor{block title alerted}{bg=yellow!20, bg=alerted text.fg!20}
\setbeamercolor{block body}{bg=structure!10}
\setbeamercolor{block title}{bg=structure!20}
\setbeamercolor{block body example}{bg=green!10}
\setbeamercolor{block title example}{bg=green!20}

% Information to be included in the title page:
\title{Title of the Document}
\subtitle{Subtitle}

\author{Author Name}
% Para a indicação de mais de um autor
% {A.~B.~Arthur\inst{1} \and J.~Doe\inst{2}}

\institute{Institution Name}
% Para indicação de mais de uma instituição de ensino.

% \institute[VFU]
% {
%   \inst{1}%
%   Faculty of Physics\\
%   Very Famous University
%   \and
%   \inst{2}%
%   Faculty of Chemistry\\
%   Very Famous University
% }

\date{\today}

\begin{document}

% Slide inicial com título
\frame{\titlepage}

% Índice
%\begin{frame}
%    \frametitle{Index}
\frame{\tableofcontents}
%\end{frame}

% Exibido, no começo de cada seção, o tópico corrente
% \AtBeginSection[]
% {
%   \begin{frame}
%     \frametitle{New Topic}
%     \tableofcontents[currentsection]
%   \end{frame}
% }

\section{sec-1}

\begin{frame}
    \frametitle{About me}
    \begin{wrapfigure}{r}{0.25\textwidth} %this figure will be at the right
        \centering
        \caption{My picture. =)}
        \includegraphics[width=0.25\textwidth]{avatar.jpeg}
        \label{fig:avatar}
    \end{wrapfigure}

    Lorem ipsum dolor sit amet, consectetur adipiscing elit.

    Curabitur eget tortor at quam ullamcorper pellentesque.
    Morbi ac enim eu ante viverra elementum id viverra justo.
    Mauris sagittis mauris sed laoreet maximus. 
    \cite{dirac}

%    \begin{figure}[h]
%        \caption{That's me. =)}
%        \includegraphics[width=0.5\textwidth, scale=0.25]{avatar.jpeg}
%    \end{faigure}

    \listoffigures

\end{frame}

\section{sec-2}

\begin{frame}
\frametitle{Frame title}

Lorem ipsum dolor sit amet, consectetur adipiscing elit.

Curabitur eget tortor at quam ullamcorper pellentesque.
Morbi ac enim eu ante viverra elementum id viverra justo.
Mauris sagittis mauris sed laoreet maximus.

Curabitur lobortis lectus vitae nisl rhoncus maximus at in lacus.
Pellentesque maximus nulla ac vehicula mollis.

\end{frame}

\section{sec-3}

\begin{frame}
    \frametitle{Highlights}
    
    In this slide, some important text will be
    \alert{highlighted} because it's important.
    Please, don't abuse it.
    
    \begin{block}{Remark}
    Sample text. Remark.
    \end{block}
    
    \begin{alertblock}{Important}
    Sample text in red box. Alert.
    \end{alertblock}
    
    \begin{examples}
    Sample text in green box. The title of the block is ``Examples".
    \end{examples}

    See my avatar: \ref{fig:avatar}

\end{frame}

\frame{\printbibliography}

\end{document}