\documentclass{beamer}

%encoding
%--------------------------------------
\usepackage[T1]{fontenc}
\usepackage[utf8]{inputenc}
%--------------------------------------

%Portuguese-specific commands
%--------------------------------------
\usepackage[portuguese]{babel}
%--------------------------------------

\usepackage{tikz}
\usepackage{pgfplots}
\pgfplotsset{compat = newest}
\usepgfplotslibrary{colormaps}

%Hyphenation rules
%--------------------------------------
\usepackage{hyphenat}
\hyphenation{mate-mática recu-perar}
%--------------------------------------

\usepackage{wrapfig}
\usepackage{graphicx}
\usepackage{biblatex}

% \usepackage[
% backend=biber,
% style=alphabetic,
% sorting=ynt
% ]{biblatex}
\addbibresource{latex-beamer-template.bib}

\graphicspath{ {../../images/} {.images} }

% \usefonttheme{structuresmallcapsserif}
\usetheme{Warsaw}
% \usecolortheme{beetle}
\usecolortheme{seagull}

% \setbeamertemplate{navigation symbols}{}
% \setbeamertemplate{mini frames}{}
\setbeamertemplate{headline}{}
% \renewcommand*{\slideentry}[6]{}
\setbeamertemplate{section in toc}[sections numbered]
\setbeamertemplate{subsection in toc}[subsections numbered]

\setbeamercolor{block body}{bg=yellow!30}
\setbeamercolor{block title}{bg=yellow!40}

\setbeamercolor{block body alerted}{bg=red!10}
\setbeamercolor{block title alerted}{bg=red!20}

\setbeamercolor{block body example}{bg=green!10}
\setbeamercolor{block title example}{bg=green!20}

% Information to be included in the title page:
\title{A Independência do Caminho na Integral do Trabalho}
\subtitle{Trabalho 1 - Grupo 16}

\author [Alex, Caio, Pedro]{
    \small Alex Campbell e Souza - Engenharia de Sistemas \\ 
    Caio Lucas Gomes Silva - Matemática \\ 
    Pedro Mansur Gamarano - Matemática
}

\institute[]{
    \large UFMG \\
    \footnotesize Universidade Federal de Minas Gerais \\
    \small Fundamentos de Eletromagnetismo
}


\date{\today}

\begin{document}

% Slide inicial com título
\frame{\titlepage}

% Índice
%\begin{frame}
%    \frametitle{Index}
\frame{\tableofcontents}
%\end{frame}

% Exibido, no começo de cada seção, o tópico corrente
% \AtBeginSection[]
% {
%   \begin{frame}
%     \frametitle{New Topic}
%     \tableofcontents[currentsection]
%   \end{frame}
% }
\section{Introdução}
\begin{frame}
    \frametitle{Introdução}
    O metodo da idependencia do caminho  é possivel apenas em  campos conservativos. Um campo é conservativo quando ele é obtido atraves do calculo do vetor gradiente de alguma função. 
    O campo eletrico é um exemplo de campo conservativo.


%     \begin{tikzpicture}
%         \begin{axis}[
%         xmin = -4, xmax = 4,
%         ymin = -4, ymax = 4,
%         zmin = 0, zmax = 1,
%         axis equal image,
%         xtick distance = 1,
%         ytick distance = 1,
%         view = {0}{90},
%         scale = 1.25,
%         title = {\bf Vector Field $F = [-y,x]$},
%         xlabel = {$x$},
%         ylabel = {$y$},
%         colormap/viridis,
%         colorbar,
%         colorbar style = {
%         ylabel = {Vector Length}
%         }
%         ]
%         \addplot3[
%         point meta = {sqrt(x^2+y^2)},
%         quiver = {
%         u = {-y/sqrt(x^2+y^2)},
%         v = {x/sqrt(x^2+y^2)},
%         scale arrows = 0.25,
%         },
%         quiver/colored = {mapped color},
%         -stealth,
%         domain = -4:4,
%         domain y = -4:4,
%         ] {0};
%         \end{axis}
%     \end{tikzpicture}
% \end{frame}

% \begin{frame}
%     \frametitle{About me}
%     \begin{wrapfigure}{r}{0.25\textwidth} %this figure will be at the right
%         \centering
%         \caption{My picture. =)}
%         \includegraphics[width=0.25\textwidth]{avatar.jpeg}
%         \label{fig:avatar}
%     \end{wrapfigure}
% 
%     Lorem ipsum dolor sit amet, consectetur adipiscing elit.
% 
%     Curabitur eget tortor at quam ullamcorper pellentesque.
%     Morbi ac enim eu ante viverra elementum id viverra justo.
%     Mauris sagittis mauris sed laoreet maximus. 
%     \cite{dirac}
% 
% %    \begin{figure}[h]
% %        \caption{That's me. =)}
% %        \includegraphics[width=0.5\textwidth, scale=0.25]{avatar.jpeg}
% %    \end{faigure}
% 
%     \listoffigures
% 
% \end{frame}
% 
%\section{sec-2}

\begin{frame}
    \frametitle{Introdução}
    A independencia do caminho diz que, em campos conservativos, quaisquer integrais de linha que possuem os mesmos pontos inicial e final resultam em um mesmo valor, independente da curva 
    entre eles. Utilizando o trabalho como exemplo:
    
    \begin{equation}
    \int_{a}^b f(x)dx
    \end{equation}


%    This agent thus does work at a rate $V_1I_1=-{\cal E}_1I_1$. The total work 
%    that it does is
%    \begin{equation}
%        W_1=-\int dtI_1{\cal E}_1=\frac{I_1}{c}\int dt
%    \end{equation}
%    where $F_1$ is the total change of the flux through the loop. We may 
%    also express our result as
% 
% \begin{equation}
%     a^{2}= b^{2}+c^{2}
% \end{equation}

\end{frame}

\section{sec-3}

\begin{frame}
    \frametitle{Highlights}
    
    In this slide, some important text will be
    \alert{highlighted} because it's important.
    Please, don't abuse it.
    
    \begin{block}{Genérico}
    Bloco ``Genérico"
    \end{block}
    
    \begin{alertblock}{Importante}
    Bloco amarelo para ``Atenção".
    \end{alertblock}
    
    \begin{examples}{Boas Práticas}
    Bloco verde para ``Boas Práticas".
    \end{examples}

\end{frame}

\frame{\printbibliography}

\end{document}